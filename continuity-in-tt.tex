\documentclass[a4paper]{article}

\usepackage[T1]{fontenc}

\usepackage[usenames, dvipsnames]{xcolor}
\usepackage[colorlinks]{hyperref}

\usepackage{mathtools}
\usepackage{amsmath}
\usepackage{amsfonts}
\usepackage{fullpage}
\usepackage[parfill]{parskip}

\usepackage{graphicx}
\setlength{\parindent}{0pt}

\title{Bibliography of Continuity in Type Theory}
\author{Bruno Rocha da Paiva \and Vincent Rahli \and Ayberk Tosun}

\hypersetup{
  colorlinks = true,
  linktoc    = page,
  citecolor  = MidnightBlue,
  linkcolor  = MidnightBlue,
  pdfauthor  = {Ayberk\ Tosun},
  pdftitle   = {Bibliography of Continuity in Type Theory},
}

\usepackage[backend=biber, style=alphabetic]{biblatex}
\addbibresource{references.bib}

\begin{document}

\maketitle

Longley~\cite{longley-not-a-functional-program-1999} pioneered the idea of using
effects to compute moduli of continuity.

Rahli and Bickford~\cite{rahli-bickford-mscs-2018} applied Longley's method to
(computational) type theory, extending the idea of writing such effectful
programs to viewing such programs as realisers of continuity principles in type
theory.

Coquand and Jaber~\cite{coq-jaber-forcing-2012} proved that MLTT-definable
functions on the Cantor space are uniformly continuous using forcing.

Ghani, Hancock, and Pattinson~\cite{ghp-continuous-functions-2006} started the
study of \emph{inductive continuity principle}.

Escard\'o~\cite{mhe-effectful-forcing-2013} used a \textbf{dialogue tree}
translation for computing moduli of continuity of System~T-definable functions.

Baillon, Mahboubi, and P\'edrot~\cite{bmp-pythia-2022} externally validated a
continuity principle for a simple \textbf{intensional type theory} with
restricted dependent elimination.

\newpage
\printbibliography

\end{document}
